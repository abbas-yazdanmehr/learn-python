%%% you can find and change template keywords by 
%%% searching "yyy" all of this template keywords 
%%% starts by "yyy".

\documentclass[a4paper, 12pt]{article}


%%% packages %%%
\usepackage[T1]{fontenc}
\usepackage{tgbonum}
\usepackage{tabto}
\usepackage[top=2cm, bottom=2cm, right=2cm, left=2cm]{geometry}
\usepackage{xcolor}
\usepackage{proof}
\usepackage{amsmath}
\usepackage{amssymb}
\usepackage{tabularx}
\usepackage{ulem}
\usepackage{listings}
\usepackage{hyperref}
\usepackage{graphicx}
\usepackage[english]{babel}
\usepackage{amsthm}


%%% defines %%%
\def\pfbox % new experimental version (DEK, November 88)
{{\ooalign{\hfil\lower.06ex % a smiley face
\hbox{$\scriptscriptstyle\frown$}\hfil\crcr
 \hfil\lower.7ex\hbox{\"{}}\hfil\crcr
 \mathhexbox20D}}}

%%% newcounter %%%
\newcounter{stepcounter}
 
%%% newenvironment %%%


%%% new theoremm %%%
\theoremstyle{definition}
\newtheorem{definition}{Definition}[section]
\newtheorem{theorem}{Note}[section]

%%% definecolors %%%
\definecolor{dkgreen}{rgb}{0,0.6,0}
\definecolor{gray}{rgb}{0.5,0.5,0.5}
\definecolor{mauve}{rgb}{0.58,0,0.82}
\definecolor{backcolour}{rgb}{0.95,0.95,0.92}
\definecolor{blue1}{rgb}{0, 0, 0.55}
\definecolor{green1}{rgb}{0, 0.4, 0}
\definecolor{red1}{rgb}{0.55, 0, 0}
\definecolor{purple1}{rgb}{0.3, 0, 0.5}

%%% newcommand %%%
\newcommand{\noticon}{\includegraphics[width=15pt]{constants/note.png}}
\newcommand{\deficon}{\includegraphics[width=20pt]{constants/def.png}}
\newcommand{\piticon}{\includegraphics[width=25pt]{constants/pitfall.png}}
\newcommand{\stepicon}{\includegraphics[width=25pt]{constants/step.png}}
\newcommand{\cmdicon}{\includegraphics[width=20pt]{constants/terminal.png}}

\newcommand{\mynote}[1]{\vspace{10pt} 
\noindent
  \hspace{0.5em}\raisebox{-0.3em} \noticon \quad \textbf{\textcolor{blue1}{#1}} 
}

\newcommand{\mydef}[1]{\vspace{10pt} 
\noindent
  \hspace{0.5em}\raisebox{-0.3em} \deficon \quad \textit{\textbf{\textcolor{blue1}{#1}}}
\\[5pt]}

\newcommand{\mypit}[1]{\vspace{10pt} 
\noindent
  \hspace{0.5em}\raisebox{-0.4em} \piticon \quad \textbf{\textcolor{red1}{#1!}} 
\\[5pt]}

\newcommand{\mystepbystep}[1]{\setcounter{stepcounter}{1}
\vspace{10pt} 
\noindent
  \hspace{0.5em}\raisebox{-0.5em} \stepicon \quad {\large \textbf{\textcolor{blue1}{#1}}}
\\ \\}

\newcommand{\mystep}[1]{
    \noindent
    \begin{tabular}{p{0.1\textwidth} p{0.85\textwidth}}
      {\Large \thestepcounter}& #1
    \end{tabular} 
    \addtocounter{stepcounter}{1} \\
}

\newcommand{\mycommand}[1]{
\vspace{10pt} 
\noindent
  \hspace{0.5em}\raisebox{-0.5em} \cmdicon \quad {\textbf{\textcolor{blue1}{#1}}}
}

\newcommand{\mycodeinput}[1]{\noindent \lstinputlisting[firstline=0, lastline=100, caption=#1, captionpos=b]{#1}}

%%% package settings %%%
\lstset{
  backgroundcolor=\color{backcolour},
  frame=tblr,
  language=Python,
  aboveskip=3mm,
  belowskip=3mm,
  showstringspaces=false,
  columns=flexible,
  basicstyle={\small\ttfamily},
  numbers=none,
  numberstyle=\tiny\color{gray},
  keywordstyle=\color{blue},
  commentstyle=\color{dkgreen},
  stringstyle=\color{mauve},
  breaklines=true,
  breakatwhitespace=true,
  tabsize=3
}



%%% begin document %%%
\begin{document}

\title{learn-python}
\author{Abbas Yazdanmehr \\ \texttt{abbas.yazdanmehr1@gmail.com}}
\date{last update: \today}
\maketitle

\newpage

\tableofcontents

\newpage

\section{Python Basics}

\mydef{Python}{Python is a scripting programming language that means it
has interpreter instead of compiler.}


\subsection{Run Python App}

\mycommand{Run Python file and continue file in scripting space}
\begin{lstlisting}
$ python -i <file_name>.py

----- ex -----
D:\learn\learn-python\doc>python -i main.py
code output: 21
>>> i
21  
>>> i**2
441 
>>> exit()
\end{lstlisting}

\mycommand{Save app output as a \textit{.txt} file in cmd}
\begin{lstlisting}
$ Python <app_name> > <output_file_name>.txt

----- ex -----
D:\learn\learn-python\doc>python main.py > output.txt

D:\learn\learn-python\doc>more output.txt 
code output: 21
\end{lstlisting}

\mypit{Pay attention to Python's operator priority}
See this:
  \begin{lstlisting}
  >>> 3 + 5 % 4 == 0
  False
  >>> (3 + 5) % 4 == 0
  True
  \end{lstlisting}

\subsection{Python Virtual Environment}

\mycommand{Create - Activate - Deactivate in cmd}
\begin{lstlisting}
-- create
$ python -m venv <virtual_environment_name>

-- activate
$ <virtual_environment_name>\Scripts\activate.bat

-- deactivate
$ deactivate

----- ex -----
D:\dir/b
code
learn
library

D:\python -m venv prjenv

D:\dir/b
code
learn
library
prjenv

D:\prjenv\Scripts\activate

(prjenv) D:\pip install lightweight
Collecting lightweight
  Downloading lightweight-1.0.0.dev52-py3-none-any.whl (256 kB)
     ------------------------------------ 256.9/256.9 KB 450.9 kB/s eta 0:00:00
...

(prjenv) D:\pip freeze
MarkupSafe==2.1.2
mistune==0.8.4
...

(prjenv) D:\deactivate
D:\
\end{lstlisting}

\mynote{Python Root packages usual path in windows}
\begin{verbatim}
C:\Users\User\AppData\Local\Packages\PythonSoftwareFoundation.Python.3.9\
_qbz5n2kfra8p0\LocalCache\local-packages\Python39\site-packages\
\end{verbatim}

\mycommand{See all installed packages in current python environment}
\begin{lstlisting}
$ pip freeze

----- ex ------
C:\Users\User>pip freeze
contourpy==1.0.7
cycler==0.11.0
fonttools==4.38.0
\end{lstlisting}

\mycommand{Save all package names in \textit{requirements.txt}}
\begin{lstlisting}
$ pip freeze > requirements.txt

----- ex ------
C:\Users\User>pip freeze > requirements.txt

C:\Users\User>more requirements.txt
contourpy==1.0.7
cycler==0.11.0
fonttools==4.38.0
\end{lstlisting}

\mycommand{Delete Packages}
\begin{lstlisting}
-- first save all package names in "requirements.txt"

-- one by one	 
$ pip uninstall -r requirements.txt

-- all			 
$ pip uninstall -r requirements.txt -y 

----- ex ------
C:\Users\User>pip freeze
contourpy==1.0.7
cycler==0.11.0
fonttools==4.38.0

C:\Users\User>pip freeze > requirements.txt

C:\Users\User>pip freeze

C:\Users\User>
\end{lstlisting}

\mystepbystep{Connect Python app to Google Sheet}
\mystep{first}
\mystep{second}
\mystep{third}

\newpage

\section{Python Packages}


\end{document}